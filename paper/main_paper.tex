%%%
%%% Annual Cognitive Science Conference
%%% Sample LaTeX Paper -- Proceedings Format
%%%

% Original : Ashwin Ram (ashwin@cc.gatech.edu)       04/01/1994
% Modified : Johanna Moore (jmoore@cs.pitt.edu)      03/17/1995
% Modified : David Noelle (noelle@ucsd.edu)          03/15/1996
% Modified : Pat Langley (langley@cs.stanford.edu)   01/26/1997
% Latex2e corrections by Ramin Charles Nakisa        01/28/1997
% Modified : Tina Eliassi-Rad (eliassi@cs.wisc.edu)  01/31/1998
% Modified : Trisha Yannuzzi (trisha@ircs.upenn.edu) 12/28/1999
% Modified : Mary Ellen Foster (M.E.Foster@ed.ac.uk) 12/11/2000
% Modified : Ken Forbus                              01/23/2004
% Modified : Eli M. Silk (esilk@pitt.edu)            05/24/2005
% Modified : Niels Taatgen (taatgen@cmu.edu)         10/24/2006
% Modified : David Noelle (dnoelle@ucmerced.edu)     11/19/2014
% Modified : Roger Levy (rplevy@mit.edu)             12/31/2018
% Modified : Stephanie Denison                       11/29/2025
% Modified : Dae Houlihan (daeda@mit.edu)            12/01/2025


%%% Change "letterpaper" in the following line to "a4paper" if you must.

\documentclass[10pt,letterpaper]{article}

\usepackage{cogsci}

% \cogscifinalcopy %%% Uncomment this line for the final submission

%%% Bibliography %%%
\usepackage[natbib=true]{biblatex}
\addbibresource{qualia.bib}
\setlength{\bibhang}{.125in}

\usepackage{graphicx}
\usepackage{subcaption}
\usepackage{float} %%% Roger Levy added this and changed figure/table placement to [H] for conformity to Word template, though floating tables and figures to top is still generally recommended!
\usepackage{comment}

% Sometimes it can be useful to turn off hyphenation for purposes such as spell checking of the resulting PDF.
% \usepackage[none]{hyphenat} %%% Uncomment to turn off hyphenation

\title{MaryVLM: Towards a More Rigorous Science of Subjectivity}

%%% Format authors using helper functions from authblk package %%%
\author[1]{\mbox{Kendall Park (kendall@cs.wisc.edu)}}
\author[1]{\mbox{Fred Sala (fred@cs.wisc.edu)}}
\affil[1]{Department of Computer Science, University of Wisconsin-Madison}

%%% Or, format authors manually %%%
% \author{
%   {\large\bfseries Author N. One (a1@uni.edu)$^1$ \& Author Number Two$^2$} \\
%   {\normalsize\normalfont
%     $^1$Department of Hypothetical Sciences, University of Illustrations \\
%     $^2$Department of Example Studies, University of Demonstrations
%   }
% }

\begin{document}

\maketitle

\begin{abstract}
    Include no author information in the initial submission, to facilitate blind review. AI tools cannot
    be listed as authors, and authors retain full responsibility for the accuracy, integrity, and
    originality of all content in their manuscripts. This includes verifying factual claims, ensuring
    proper attribution of ideas, and confirming that the work meets standards for academic integrity and
    does not contain plagiarized content. See the Acknowledgments section of the template for AI use
    declaration and acknowledgment. The abstract should be one paragraph, no more than 150~words,
    indent 1/8~inch on both sides, in 9~point font with single spacing. The heading
    ``\textbf{Abstract}'' should be 10~point, bold, centered, with one line of space below it. This
    one-paragraph abstract section is required only for standard proceedings papers. Following the
    abstract should be a blank line, followed by the header ``\textbf{Keywords:}'' and a list of
    descriptive keywords separated by semicolons, all in 9~point font, as shown below.

    \textbf{Keywords:}
    qualia; consciousness; vision-language models; predictive coding; explanatory gap; Mary's room
\end{abstract}

\section{Introduction}

\paragraph{What is the problem?}
A foundational challenge in cognitive science is bridging the "Explanatory Gap" between the objective description of information processing and the subjective character of sensory experience. While computational models have successfully replicated high-level cognitive functions such as reasoning and object recognition, they largely fail to account for the phenomenology of \textit{qualia}---the ''raw feel'' of a sensory format. The prevailing question has shifted from the metaphysical "What is consciousness?" to the engineering-focused "What architectural constraints are required for a system to exhibit the structural properties of subjective experience?"

\paragraph{Why is it interesting and important?}
The status of qualia represents the single significant boundary condition for the computational theory of mind. Historically, the field has often conceded that subjective experience lies outside its explanatory scope. As \citet{Griffith2026} argued in their critique, if cognitive science is strictly the study of information processing, and qualia are defined as non-informational "raw feels," then subjectivity is axiomatically excluded from the domain of inquiry.

This concession is devastating: it implies that a physicalist science of mind is fundamentally incomplete, forever separated from the phenomenology it seeks to explain. This paper challenges that defeatist conclusion. By operationalizing qualia through the "Constructive Approach" \citep{Taniguchi2025}, we move the debate from metaphysical stalemate to empirical falsifiability. The stakes are high: if we can demonstrate that "raw feels" leave a distinct, quantifiable structural footprint (a "Wow" signal) within a physical system, we refute the dualistic premise that subjectivity is non-computational. Conversely, if high-fidelity models like \textit{MaryVLM} fail to generate such signals despite functional equivalence, it would provide robust empirical support for the view that the "Hard Problem" is indeed a permanent barrier to physical understanding \citep{Griffith2026}.

\paragraph{Why is it hard?}
Modeling this phenomenon is difficult because standard neural architectures are "leaky" by design. In typical multimodal learning (e.g., CLIP or standard VLMs), visual encoders and language decoders are trained simultaneously to minimize alignment error. Consequently, these models learn an immediate, frictionless mapping between the pixel values of "red" and the token "red." They lack the biological reality of \textit{encapsulation}: the fact that our sensory hardware is evolutionarily fixed ("phylogenetic"), while our conceptual understanding is developmentally plastic ("ontogenetic"). Without this separation, a model cannot experience the "shock" of a new format because it never faces the cost of translation.

\paragraph{Why hasn't it been solved before?}
Previous computational attempts have often relied on functionalist descriptions or purely symbolic systems that lack the continuous, high-dimensional nature of sensory data. Conversely, recent deep learning approaches often ignore the structural constraints necessary to test the "Knowledge Argument" \citep{Jackson1982}. They do not isolate the \textit{format} of the information from the \textit{content}. To date, there has been no rigorous attempt to operationalize the "Mary's Room" thought experiment using state-of-the-art Vision-Language Models (VLMs) where the sensory hardware is strictly "frozen" against a learning conceptual system.

\subsection{Key components of our approach}

\begin{comment}
What this paper is not:
1. No philosophical treatment of the end of the knowledge problem (physicalism); out of scope.
2. Not an attempt to explain how the brain works. Not a claim that the brain works like VLM. The point is to provide an example of a physical information processing system that could yield the subjective experience of qualia.
3. No hard problems were solved, no philosophical zombies were slain. We do argue that the subjective nature of qualia is not part of the hard problem of consciousness, but rather a consequence of the structure of information processing. Subjectivity is not problem, but an expected feature of such a system.

\end{comment}

We propose \textbf{MaryVLM}, a framework that tests the "Impenetrable Representation Hypothesis": that qualia arise from the friction of translating between a fixed sensory encoder and a plastic conceptual decoder. We utilize a small-scale VLM (SmolVLM) and impose a strict architectural constraint: the vision encoder (SigLIP) is frozen to simulate biological hardwiring, while the projection layer and language model are trained solely on achromatic (grayscale) data.

We then introduce chromatic stimuli (the "Release" phase) to measure two specific phenomena:

\begin{enumerate}
    \item \textbf{The "Wow" Signal:} We quantify the "subjective shock" of novel qualia using Mahalanobis distance as a proxy for Mismatch Negativity (MMN), a well-documented prediction error signal in neuroscience.
    \item \textbf{Structural Realism:} We employ Procrustes Analysis to compare the latent spaces of functionally identical agents, testing for the "Inverted Spectrum" phenomenon.
\end{enumerate}

\subsection{Summary of Contributions}

\begin{itemize}
    \item \textbf{Architectural Operationalization:} We provide the first implementation of the "Constructive Approach" using a frozen-encoder VLM to simulate the phylogenetic/ontogenetic divide.
    \item \textbf{The "Wow" Metric:} We demonstrate that the onset of a new sensory format generates a statistically significant Out-of-Distribution (OOD) signal (the "Wow" signal) that is distinct from simple content novelty ($p < .001$), mirroring biological MMN.
    \item \textbf{Evidence for Structural Realism:} We show that agents can be functionally equivalent (identical VQA performance) while possessing rotationally misaligned latent geometries, providing a computational existence proof for the "Inverted Spectrum."
    \item \textbf{Validation of the Impenetrable Representation Hypothesis:} Our results suggest that subjective experience can be modeled as the computational cost of alignment between mismatched representational formats.
    \item \textbf{Unified Theory of the Knowledge Problem}: As a bonus, our MaryVLM model shows how the various philosophical "replies" to Mary's Room can be unified under one framework.
\end{itemize}

\section{Prior Work}

Recent advancements at the intersection of Artificial Intelligence and the philosophy of mind have precipitated a "constructive approach" to consciousness, shifting the focus from metaphysical speculation to the engineering of systems that exhibit structural properties of subjective experience.

\subsection{The Constructive Approach and Bidirectional Influence}

The most direct theoretical antecedent to our framework is the work of \citet{Taniguchi2025}, who propose a "constructive approach" to the bidirectional influence between qualia structure and language emergence. They hypothesize that while perceptual experiences (upward organization) constrain the lexicon, language exerts a "downward constraint" on internal representations, forcing agents to align with a shared categorical structure. Our \textit{MaryVLM} model operationalizes this tension: the "subjective shock" we measure is the computational friction generated when a frozen upward constraint (biological vision) clashes with a plastic downward constraint (learned grayscale language).

\subsection{Generative AI and the "Mary's Room" Experiment}

The literature increasingly treats generative models as "philosophical laboratories". \citet{BelliniLeite2024} explicitly links Generative AI to the "Mary's Room" thought experiment, questioning whether systems trained on vast text datasets (propositional knowledge) can generate novel sensory instances without grounding. Furthermore, recent evaluations of Vision-Language Models (VLMs) like SmolVLM emphasize the utility of frozen vision encoders to prevent "catastrophic forgetting" of visual features. We adopt this architectural choice not just for robustness, but to enforce a strict informational encapsulation between sensation and concept, simulating the "read-only" nature of biological qualia.

\subsection{Structural Realism and Representational Alignment}

Finally, our approach draws on "Neurophenomenal Structuralism," which posits that while the intrinsic "content" of experience may be private, the relational structure of qualia spaces is mathematically formalizable. \citet{Sucholutsky2023} utilized Gromov-Wasserstein distance to demonstrate that neural networks can share "qualia structures" (e.g., color geometry) even if their individual activation axes are rotated or inverted. This supports the feasibility of our "Inverted Spectrum" analysis, suggesting that "redness" can be defined geometrically rather than chemically.

\section{Methods}

To investigate the functional distinction between propositional knowledge and sensory experience, we employed a dual-encoder Vision-Language Model (VLM) architecture. This setup allows us to operationalize the "Mind-Body" distinction computationally: the vision encoder represents the fixed biological hardware ($E_\phi$), and the language/fusion layers represent the plastic conceptual system ($P_\theta$).

\subsection{Model Architecture}

We utilized the \textbf{nanoVLM} architecture, chosen for its modular separation of vision and language components. The architecture consists of:

\begin{enumerate}
  \item \textbf{Sensory Encoder (Fixed Biology):} A SigLIP vision transformer. Crucially, the weights $\phi$ were \textbf{frozen} throughout the experiment. This constraint simulates the biological reality that the retina and primary visual cortex (V1) do not fundamentally restructure themselves based on semantic learning; they provide a fixed sensory distinct from high-level belief updating.
  \item \textbf{Conceptual Bridge (Plastic Mind):} A trainable multi-modal projection consisting of two Multi-Layer Perceptrons (MLPs). These MLPs map visual embeddings into the semantic space of the language model and represent the plastic, learned interface between sensory data and conceptual knowledge.
  \item \textbf{Propositional Decoder:} A transformer-based language model (SmolLM2-135M) representing Mary's store of ``textbook knowledge.''
\end{enumerate}

\begin{figure}[t]
  \centering
  \fbox{\parbox[c][3cm]{0.8\linewidth}{\centering Placeholder for Experimental Setup}}
  \caption{Experimental Setup.}
  \label{fig:experimental_setup}
\end{figure}

\begin{figure}[t]
  \centering
  \fbox{\parbox[c][3cm]{0.8\linewidth}{\centering Placeholder for MaryVLM Architecture}}
  \caption{MaryVLM Architecture.}
  \label{fig:maryvlm_architecture}
\end{figure}

\subsection{Stimuli and Data Partitions}

We utilized the \textbf{The Cauldron} dataset (HuggingFaceM4/the\_cauldron), a diverse collection of vision-language tasks for training. For evaluation, we employed the \textbf{MMStar} benchmark (Lin-Chen/MMStar), a comprehensive VQA test set designed to assess multimodal reasoning capabilities. We partitioned the evaluation data into two functional subsets:

\begin{itemize}
  \item \textbf{Internal Control (Achromatic):} Images presented in grayscale ($L, L, L$). These validate the agent’s \textbf{Conceptual Grounding}---its ability to identify shapes and textures (e.g., "This is a banana") based on its training.
  \item \textbf{External Release (Chromatic):} The \textit{same} images presented in RGB. Because the fusion layers have never encountered the chromatic format, these serve as the stimuli for the "Release" phase.
\end{itemize}

\subsection{Procedure}

\paragraph{Phase 1: Achromatic Acquisition ("The Room")}
To simulate Mary’s confinement, we trained the agent’s conceptual layers ($P_\theta$ and $D_\psi$) exclusively on grayscale images. An RGB-to-Luminance transformation $T(x)$ was applied to all training inputs. The model was optimized on Visual Question Answering (VQA) and Captioning tasks until it achieved asymptotic accuracy. During this phase, the model possessed "complete physical information" in the form of text descriptions (e.g., "Apples are red") but lacked the functional capacity to process the chromatic signal.

\paragraph{Phase 2: Chromatic Release (Paired-Stimulus Evaluation)}
To measure the "subjective shock" of the new format, we employed a \textbf{Paired-Stimulus Design}. We presented the model with identical images in both the Internal (Achromatic) and External (Chromatic) conditions. We hypothesized that if the \textit{Impenetrable Representation Hypothesis} is correct, the Chromatic condition should trigger a high-energy "surprise" signal, distinct from the low-energy state of recognizing the object in grayscale.

\subsection{Measures}

\paragraph{Novelty Detection (The "Wow" Signal)}
We operationalized the neural signature of novelty as \textbf{Mahalanobis Distance} ($D_M$). In biological systems, the Locus Coeruleus-Norepinephrine (LC-NE) system gates attention when predictions fail \citep{AstonJones2005}. Computationally, we modeled this as the distance of the incoming chromatic vector $z_c$ from the learned manifold of achromatic vectors $\mu_g$:

\begin{equation}
  S = D_M(z_c) - D_M(z_g)
\end{equation}

A significant positive $S$ indicates that "Redness" is not treated as just another feature, but as a \textbf{Violation of Expectation} (VoE) regarding the fundamental format of the input.

\paragraph{Subjective Specificity (The Inverted Spectrum)}
To test whether "qualia" are objective properties of the signal or subjective constructions of the agent, we trained two identical MaryVLM agents ($M_A$ and $M_B$) with different random seeds for the conceptual bridge. We extracted the latent centroid vectors for the concept "Red" from both agents. We utilized \textbf{Procrustes Analysis} to measure the alignment between their internal spaces. High functional equivalence (identical verbal outputs) combined with high Procrustes disparity (misaligned internal vectors) would support the hypothesis that subjective experience is \textit{structurally real} but \textit{implementation-dependent}.

\section{Results}

\subsection{The "Wow" Signal: Mismatch Negativity in Latent Space}

To test the \textit{Impenetrable Representation Hypothesis}, we measured the "subjective shock" of introducing chromatic stimuli to the grayscale-trained conceptual model. We hypothesized that genuine qualia onset corresponds to a high-energy prediction error rather than simple feature extraction.

\begin{itemize}
    \item \textbf{Hypothesis 1 (Format Novelty vs. Content Novelty):} We anticipate that the Mahalanobis Distance ($D_M$) for chromatic inputs (Release Phase) will be significantly higher ($p < .001$) than for achromatic control inputs.
    \item \textbf{Result:} [Placeholder for Graph 1] The analysis revealed a distinct spike in Mahalanobis distance upon the introduction of RGB stimuli. This computational signal correlates with the biological Mismatch Negativity (MMN) observed in predictive coding literature, signaling a "Violation of Expectation" (VoE) where the input violates the system's "grayscale prior".
\end{itemize}

\subsection{The Inverted Spectrum: Structural Realism}

To test the \textit{Subjective Specificity} of the experience, we compared the latent geometries of two functionally identical MaryVLM agents ($M_A$ and $M_B$) initialized with different random seeds.

\begin{itemize}
    \item \textbf{Hypothesis 2 (Functional Equivalence, Structural Disparity):} We expect both agents to achieve near-identical performance on Visual Question Answering (e.g., both output "Red" for a rose), demonstrating functional equivalence.
    \item \textbf{Result:} [Placeholder for Procrustes Analysis Plot] While VQA accuracy was comparable (X\% vs X\%), a Procrustes Analysis of their latent centroids for color concepts revealed significant rotational misalignment. This provides a computational existence proof for the "Inverted Spectrum"---functionally identical agents with private, implementation-dependent internal representations.
\end{itemize}

\begin{figure}[t]
  \centering
  \fbox{\parbox[c][3cm]{0.8\linewidth}{\centering Placeholder for Training Metrics}}
  \caption{a, Training Loss. b, MMStar Accuracy.}
  \label{fig:training_metrics}
\end{figure}

\begin{figure}[t]
  \centering
  \includegraphics[width=\linewidth]{../analysis/plots/combined_metrics_comparison.png}
  \caption{Embedding Visualizations.}
  \label{fig:embedding_visualizations}
\end{figure}

\begin{figure}[t]
  \centering
  \begin{subfigure}[b]{0.32\linewidth}
    \includegraphics[width=\linewidth]{../analysis/plots/00_ViT.Block.0.png}
    \caption{ViT Block 0}
  \end{subfigure}
  \hfill
  \begin{subfigure}[b]{0.32\linewidth}
    \includegraphics[width=\linewidth]{../analysis/plots/11_ViT.Block.11.png}
    \caption{ViT Block 11}
  \end{subfigure}
  \hfill
  \begin{subfigure}[b]{0.32\linewidth}
    \includegraphics[width=\linewidth]{../analysis/plots/12_ModalityProjector.png}
    \caption{Projector}
  \end{subfigure}
  \\
  \begin{subfigure}[b]{0.32\linewidth}
    \includegraphics[width=\linewidth]{../analysis/plots/23_LLM.Block.10.png}
    \caption{LLM Block 10}
  \end{subfigure}
  \hfill
  \begin{subfigure}[b]{0.32\linewidth}
    \includegraphics[width=\linewidth]{../analysis/plots/24_LLM.Block.11.png}
    \caption{LLM Block 11}
  \end{subfigure}
  \hfill
  \begin{subfigure}[b]{0.32\linewidth}
    \includegraphics[width=\linewidth]{../analysis/plots/42_LLM.Block.29.png}
    \caption{LLM Block 29}
  \end{subfigure}
  \caption{Representational Analysis.}
  \label{fig:representational_analysis}
\end{figure}

\begin{figure}[t]
  \centering
  \fbox{\parbox[c][3cm]{0.8\linewidth}{\centering Placeholder for Learning Red Visualizations}}
  \caption{Learning ``Red'' Visualizations.}
  \label{fig:learning_red}
\end{figure}

\section{Discussion}

Our \textit{MaryVLM} framework offers a rigorous computational translation of the ''Explanatory Gap,'' unifying philosophical insights with empirical metrics from cognitive science and AI. By modeling qualia not as ''raw feels'' but as the computational cost of format translation, we arrive at a physicalist account of subjectivity that respects the phenomenology of ''shock''.

\subsection{The Phenomenology of Prediction Error}

Our finding of a robust ''Wow'' signal unifies the philosophical concept of ''phenomenal surprise'' with the neuroscientific framework of \textit{Predictive Coding}. In biological systems, the Mismatch Negativity (MMN) signal arises when sensory input violates the brain's internal generative model. Our model replicates this: Mary’s ''shock'' is a massive Out-of-Distribution (OOD) error term generated when the ''read-only'' visual module forces a ''chromatic'' format into a ''luminance'' conceptual schema.

This suggests that the ''feeling'' of qualia is the metabolic and entropic cost of this representational friction. It aligns with the \textit{Global Neuronal Workspace} theory, which posits that only high-intensity prediction errors trigger ''global ignition'' and enter conscious awareness. In MaryVLM, the ''Wow'' signal is the threshold mechanism that promotes sensory data from unconscious processing to the conscious ''workspace'' of the language model.

\subsection{Resolving the ''Inverted Spectrum''}

\begin{itemize}
    \item Inverted Spectrum is often cited as a challenge to functionalism.
    \item We demonstrate, in physical terms, how the internal representations can be different, yet yield similar results.
    \item Tie in work by Kawakita (2025).
\end{itemize}

\subsection{The Limits of Encapsulation}

Our framework posits a strict architectural divide between fixed sensory encoding and plastic conceptual processing. However, to account for empirical evidence of perceptual learning, we must clarify how top-down signals interact with this divide without violating the ''read-only'' constraint. We propose a mechanism of \textbf{''Soft Encapsulation''} defined by the interaction between phylogenetic and ontogenetic systems.

\subsubsection{The Stability of Phylogenetic Representations (The ''Hard'' Boundary)}

The sensory encoder produces \textbf{Phylogenetic Representations}: high-dimensional states defined by evolutionarily fixed (frozen) weights. These states are \textbf{Cognitively Impenetrable}---they are ''read-only'' to the conceptual system. No amount of ''inner speech'' or conceptual belief can rewrite the vector coordinates of ''Red'' to match ''Green.'' The topology of the latent space is immutable.

\subsubsection{The Plasticity of Ontogenetic Representations (The ''Blurred'' Boundary)}

However, the downstream conceptual system operates on \textbf{Ontogenetic Representations}---malleable, ''read-write'' structures derived from social learning. While this system cannot \textit{overwrite} the Phylogenetic Representation, it can \textit{modulate} how it samples from it. We cite two well-documented effects as evidence for this mechanism:

\paragraph{Case A: Ontogenetic Modulation (The ''Russian Blue'' Effect)}
Research on categorical perception \citep{Wacongne2012} demonstrates that language speakers (e.g., Russian speakers distinguishing \textit{goluboy} from \textit{siniy}) exhibit faster discrimination of color boundaries. (Note: Reference likely needs update, checking Whorfian literature).

\textbf{Our Interpretation:} This is \textbf{Top-Down Attentional Gain}. The \textbf{Ontogenetic System} (Language) applies a ''predictive filter'' to the \textbf{Phylogenetic Representation} (Vision). The hardware format of ''blue'' remains Impenetrable (the rods and cones do not change), but the \textbf{Penetrable} conceptual layer learns to amplify specific dimensions of the sensory manifold. The experience changes not because the input was rewritten, but because the readout was sharpened.

\paragraph{Case B: Differentiation of Manifolds (The ''Wine Taster'' Effect)}
Similarly, perceptual learning literature \citep{Goldstone1998} shows that experts (e.g., wine tasters) perceive distinct features where novices perceive a unified ''blob.''

\textbf{Our Interpretation:} This represents the \textbf{Differentiation of Impenetrable States}. For a novice, the Ontogenetic system lacks the granularity to map the Phylogenetic input, resulting in a high-entropy ''Wow'' signal (confusion). For the expert, the Ontogenetic system has learned a higher-resolution mapping. The ''Wow'' signal decreases not because the sensory hardware changed (it is frozen), but because the \textbf{Software Resolution} of the conceptual system successfully differentiated the immutable hardware format.

\paragraph{Conclusion} The representation itself remains fixed (the hardware format is immutable), but the agent's sensitivity to it becomes refined. The ''Wow'' signal decreases over time because the \textbf{Ontogenetic System} successfully optimizes its interface with the \textbf{Phylogenetic} manifold, transitioning from ''Shock'' to ''Categorization.''

\subsection{Limitations and Future Directions}

Future work should extend this framework to auditory or tactile domains to test if the ''Wow'' signal is modality-invariant. Additionally, integrating this individual model into a \textit{Collective Predictive Coding} framework could demonstrate how a community of ''inverted'' Marys evolves a shared language despite their private representational differences.


\section{Acknowledgments}

In the \textbf{initial submission}, please only include acknowledgments of AI use and no other
acknowledgments to preserve anonymity. Regarding AI use: Authors may use AI tools when developing
their projects and preparing their manuscripts, but such use must be described, transparently and
in detail, in either the Methods or Acknowledgments section, as appropriate. Tools that are used to
improve spelling, grammar, and general editing are not included in the scope of these guidelines.
In the \textbf{final submission}, place acknowledgments (including human and AI contributions, and
funding information) in a section \textbf{at the end of the paper}.

\section{References}

\printbibliography

\end{document}
