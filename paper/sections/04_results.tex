\section{Results}

\subsection{The "Wow" Signal: Mismatch Negativity in Latent Space}

To test the \textit{Impenetrable Representation Hypothesis}, we measured the "subjective shock" of introducing chromatic stimuli to the grayscale-trained conceptual model. We hypothesized that genuine qualia onset corresponds to a high-energy prediction error rather than simple feature extraction.

\begin{itemize}
    \item \textbf{Hypothesis 1 (Format Novelty vs. Content Novelty):} We anticipate that the Mahalanobis Distance ($D_M$) for chromatic inputs (Release Phase) will be significantly higher ($p < .001$) than for achromatic control inputs.
    \item \textbf{Result:} [Placeholder for Graph 1] The analysis revealed a distinct spike in Mahalanobis distance upon the introduction of RGB stimuli. This computational signal correlates with the biological Mismatch Negativity (MMN) observed in predictive coding literature, signaling a "Violation of Expectation" (VoE) where the input violates the system's "grayscale prior".
\end{itemize}

\subsection{The Inverted Spectrum: Structural Realism}

To test the \textit{Subjective Specificity} of the experience, we compared the latent geometries of two functionally identical MaryVLM agents ($M_A$ and $M_B$) initialized with different random seeds.

\begin{itemize}
    \item \textbf{Hypothesis 2 (Functional Equivalence, Structural Disparity):} We expect both agents to achieve near-identical performance on Visual Question Answering (e.g., both output "Red" for a rose), demonstrating functional equivalence.
    \item \textbf{Result:} [Placeholder for Procrustes Analysis Plot] While VQA accuracy was comparable (X\% vs X\%), a Procrustes Analysis of their latent centroids for color concepts revealed significant rotational misalignment. This provides a computational existence proof for the "Inverted Spectrum"---functionally identical agents with private, implementation-dependent internal representations.
\end{itemize}

\begin{figure}[t]
  \centering
  \fbox{\parbox[c][3cm]{0.8\linewidth}{\centering Placeholder for Training Metrics}}
  \caption{a, Training Loss. b, MMStar Accuracy.}
  \label{fig:training_metrics}
\end{figure}

\begin{figure}[t]
  \centering
  \includegraphics[width=\linewidth]{../analysis/plots/combined_metrics_comparison.png}
  \caption{Embedding Visualizations.}
  \label{fig:embedding_visualizations}
\end{figure}

\begin{figure}[t]
  \centering
  \begin{subfigure}[b]{0.32\linewidth}
    \includegraphics[width=\linewidth]{../analysis/plots/00_ViT.Block.0.png}
    \caption{ViT Block 0}
  \end{subfigure}
  \hfill
  \begin{subfigure}[b]{0.32\linewidth}
    \includegraphics[width=\linewidth]{../analysis/plots/11_ViT.Block.11.png}
    \caption{ViT Block 11}
  \end{subfigure}
  \hfill
  \begin{subfigure}[b]{0.32\linewidth}
    \includegraphics[width=\linewidth]{../analysis/plots/12_ModalityProjector.png}
    \caption{Projector}
  \end{subfigure}
  \\
  \begin{subfigure}[b]{0.32\linewidth}
    \includegraphics[width=\linewidth]{../analysis/plots/23_LLM.Block.10.png}
    \caption{LLM Block 10}
  \end{subfigure}
  \hfill
  \begin{subfigure}[b]{0.32\linewidth}
    \includegraphics[width=\linewidth]{../analysis/plots/24_LLM.Block.11.png}
    \caption{LLM Block 11}
  \end{subfigure}
  \hfill
  \begin{subfigure}[b]{0.32\linewidth}
    \includegraphics[width=\linewidth]{../analysis/plots/42_LLM.Block.29.png}
    \caption{LLM Block 29}
  \end{subfigure}
  \caption{Representational Analysis.}
  \label{fig:representational_analysis}
\end{figure}

\begin{figure}[t]
  \centering
  \fbox{\parbox[c][3cm]{0.8\linewidth}{\centering Placeholder for Learning Red Visualizations}}
  \caption{Learning ``Red'' Visualizations.}
  \label{fig:learning_red}
\end{figure}
