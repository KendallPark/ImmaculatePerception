\section{Results}

\subsection{The "Wow" Signal: Mismatch Negativity in Latent Space}

To test the \textit{Impenetrable Representation Hypothesis}, we measured the "subjective shock" of introducing chromatic stimuli to the grayscale-trained conceptual model. We hypothesized that genuine qualia onset corresponds to a high-energy prediction error rather than simple feature extraction.

\begin{itemize}
    \item \textbf{Hypothesis 1 (Format Novelty vs. Content Novelty):} We anticipate that the Mahalanobis Distance ($D_M$) for chromatic inputs (Release Phase) will be significantly higher ($p < .001$) than for achromatic control inputs.
    \item \textbf{Result:} [Placeholder for Graph 1] The analysis revealed a distinct spike in Mahalanobis distance upon the introduction of RGB stimuli. This computational signal correlates with the biological Mismatch Negativity (MMN) observed in predictive coding literature, signaling a "Violation of Expectation" (VoE) where the input violates the system's "grayscale prior".
\end{itemize}

\subsection{The Inverted Spectrum: Structural Realism}

To test the \textit{Subjective Specificity} of the experience, we compared the latent geometries of two functionally identical MaryVLM agents ($M_A$ and $M_B$) initialized with different random seeds.

\begin{itemize}
    \item \textbf{Hypothesis 2 (Functional Equivalence, Structural Disparity):} We expect both agents to achieve near-identical performance on Visual Question Answering (e.g., both output "Red" for a rose), demonstrating functional equivalence.
    \item \textbf{Result:} [Placeholder for Procrustes Analysis Plot] While VQA accuracy was comparable (X\% vs X\%), a Procrustes Analysis of their latent centroids for color concepts revealed significant rotational misalignment. This provides a computational existence proof for the "Inverted Spectrum"---functionally identical agents with private, implementation-dependent internal representations.
\end{itemize}
