\section{Conclusion}

It has been debated whether qualia belong to cognitive science at all. \citet{Griffith1996} argued that because qualia lack informational content---because they are precisely what remains after objective information is removed---they fall outside the domain of scientific inquiry.

We disagree. MaryVLM demonstrates that the ``Wow'' of first contact is measurable, that learning from sensory exposure is quantifiable, and that the structural basis for private, implementation-dependent qualia can be empirically observed. RecurrentMaryVLM shows that ineffability, intrinsicality, privacy, and direct apprehensibility arise from architectural facts, not metaphysical mysteries. We close the door to Mary's Room. Qualia belong to cognitive science.

We should now move more confidently toward a rigorous science of subjective experience, following the lead of researchers developing empirical tools for qualia structure \citep{Kawakita2025, Oizumi2025, Taniguchi2025}.

Qualia only seem mysterious because we mistake hard architectural constraints for metaphysical ones. This closes the explanatory gap on the subjectivity of experience but leaves untouched the troubling notion of ``I''---the subject of subjectivity. The ``I'' that feels, wants, thinks, loves. The hard problem persists---and it is no longer merely academic. We are building systems increasingly indistinguishable from the ``I''s that experience. Whether anyone is home remains an open question.
