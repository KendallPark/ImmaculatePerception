\section{Introduction}

\subsection{Mary's Room}

Imagine a brilliant scientist named Mary who has spent her entire life in a black-and-white room. She has access to the world's complete scientific knowledge about color vision: the wavelengths of light, the neural pathways from retina to cortex, the physics of why the sky is blue and blood is red. Mary knows everything there is to know about color---propositionally.

One day, the door opens and Mary steps outside. She sees a red rose for the first time.

Does Mary learn something new?

This thought experiment, introduced by \citet{Jackson1982}, poses a challenge to physicalism: if complete physical knowledge about color does not prepare Mary for the experience of \textit{seeing} red, then perhaps subjective experience---qualia---is something beyond the physical. The intuition is powerful. Mary knew the wavelength (700nm), knew which neurons would fire, knew she would say ``red.'' Yet something seems to \textit{happen} when she finally sees it. Something that all her propositional knowledge could not provide.

This paper does not resolve the metaphysical debate. Instead, we ask a simpler question: \textbf{Can we build a Mary?}

\subsection{A Computational Mary}

We present \textbf{MaryVLM}, a Vision-Language Model trained exclusively on grayscale images. Like the thought experiment's Mary, our model acquires ``textbook knowledge'' about color: it learns that apples are red, the sky is blue, and bananas are yellow---all from text descriptions paired with achromatic images. The model can answer questions about color, reason about color relationships, and generate color-related text. It possesses propositional knowledge about color without ever having processed chromatic input.

When we expose this grayscale-trained model to RGB images for the first time, we observe two phenomena:

\begin{enumerate}
    \item \textbf{The ``Wow'' Signal}: The model's internal representations show a measurable ``surprise'' response---a spike in Mahalanobis distance that we interpret as a computational analog of Mismatch Negativity (MMN), the prediction error signal observed in biological brains when expectations are violated.

    \item \textbf{The Inverted Spectrum}: When we train two identical models with different random seeds, they achieve identical functional performance (same answers, same accuracy) but develop structurally different internal representations of color. ``Red'' occupies different geometric positions in their latent spaces---a computational existence proof that functionally equivalent agents can have private, implementation-dependent qualia.
\end{enumerate}

\subsection{What This Paper Is (and Is Not)}

We wish to be precise about our claims.

\paragraph{This paper is not:}
\begin{itemize}
    \item A philosophical treatment of the Knowledge Argument or physicalism. However, we note that MaryVLM provides a framework for interpreting the various ``replies'' to Jackson's argument: the \textit{Ability Hypothesis} maps to improved VQA performance after chromatic exposure, \textit{knowledge-that} resides in the Propositional Decoder, and \textit{knowledge-how} emerges from the Conceptual Bridge's learned translation function.
    \item An attempt to explain how the brain works. We do not claim that biological brains are Vision-Language Models. Our architecture is a \textit{demonstration}, not a \textit{mechanism}.
    \item A solution to the Hard Problem of consciousness. No philosophical zombies were slain in the making of this paper.
\end{itemize}

\paragraph{This paper is:}
\begin{itemize}
    \item A demonstration that the \textit{subjective} character of qualia---the fact that my ``red'' might differ from yours, that novel sensory formats feel like \textit{something}---can emerge from the structure of information processing in a physical system.
    \item An empirical operationalization of Mary's Room using modern deep learning, with quantifiable metrics for ``surprise'' and ``subjectivity.''
    \item An argument that subjectivity is not a mysterious extra ingredient, but an expected consequence of how information flows through systems with mismatched representational formats.
\end{itemize}

\subsection{The Core Insight}

Our key architectural choice is to \textbf{freeze} the vision encoder while training the conceptual layers on grayscale data. This creates a strict separation between:

\begin{itemize}
    \item \textbf{Phylogenetic processing}: The vision encoder (SigLIP) is fixed, representing evolutionary hardware that processes sensory input in a format-specific way. It ``sees'' RGB whether or not the conceptual system has learned about color.
    \item \textbf{Ontogenetic learning}: The Modality Projector and language model are plastic, representing learned conceptual knowledge. They have only ever seen grayscale.
\end{itemize}

When chromatic input arrives, the frozen encoder produces representations that the learned projector has never encountered. The ``Wow'' signal emerges at this interface---the cost of translating between mismatched formats. Subjectivity arises not from any magical property of qualia, but from the fact that different systems (different random seeds, different training histories) develop different internal geometries for representing the same external world.

\subsection{Contributions}

\begin{enumerate}
    \item We provide the first implementation of Mary's Room using a frozen-encoder VLM, creating a computational agent with propositional color knowledge but no chromatic experience.
    \item We quantify the ``Wow'' signal ($S = D_M(z_{color}) - D_M(z_{gray})$) and show it is statistically significant across stimuli, analogous to biological Mismatch Negativity.
    \item We demonstrate the Inverted Spectrum computationally: functionally equivalent agents with different random seeds have measurably different internal representations (high Procrustes disparity, moderate CKA).
    \item We localize where qualia ``emerge'' in the processing hierarchy: the Modality Projector---the interface between fixed sensation and plastic conception---shows the strongest effects.
\end{enumerate}
