\section{Introduction}

Qualia---the subjective, felt qualities of experience---remain ``central in philosophy of mind and one of the greatest obstacles in neuroscience'' \citep{Gouveia2022}. The redness of red, the painfulness of pain, the way coffee smells to you. These phenomena resist explanation in a way that threatens to place them outside scientific inquiry entirely. If qualia are truly ineffable, private, intrinsic, and directly apprehensible---as philosophers have claimed---then perhaps no physical account could ever capture them.

This paper argues otherwise. We propose that qualia only seem mysterious because we mistake hard architectural constraints for metaphysical ones.

To make this case, we build a computational Mary. Jackson's \citeyearpar{Jackson1982} thought experiment imagines a scientist who knows everything about color but has never seen it. When she finally sees red, does she learn something new? The intuition that she does has fueled decades of debate about whether subjective experience escapes physical explanation.

We present MaryVLM, a Vision-Language Model trained exclusively on grayscale images. Like Jackson's Mary, our model acquires propositional knowledge about color---it learns that apples are red and the sky is blue---without ever processing chromatic input. When we expose this model to color for the first time, we observe:

\begin{enumerate}
    \item \textbf{The ``Wow'' Signal}: A measurable surprise response---a spike in Mahalanobis distance analogous to biological Mismatch Negativity---when chromatic input violates the system's learned grayscale prior.
    \item \textbf{Learning Something New}: Continued training on chromatic input improves the model's performance on color-related tasks---demonstrating that exposure to a new sensory format confers new abilities, distinct from the immediate surprise of first contact.
    \item \textbf{Structural Basis for the Inverted Spectrum}: Two models trained identically but with different random seeds achieve equivalent functional performance while developing structurally different internal representations of color---a computational demonstration that functionally equivalent agents can have private, implementation-dependent qualia.
\end{enumerate}

These findings demonstrate that the subjective character of qualia---the fact that novel formats feel like something, that my red might differ from yours---can emerge from information processing architecture.

We then extend beyond our empirical results to propose RecurrentMaryVLM, a theoretical framework for understanding Dennett's four properties of qualia \citep{Dennett1988}. By analyzing a hypothetical architecture where a language model's output loops back as input---creating a recurrent ``inner speech'' system \citep{Vygotsky1986}---we show that ineffability, intrinsicality, privacy, and direct apprehensibility all arise from basic structural facts about where the self-reflective system sits relative to sensory representations.

\subsection{What This Paper Is (and Is Not)}

We wish to be precise about our claims.

\paragraph{This paper is not:}
\begin{itemize}
    \item An attempt to explain how the brain works. We do not claim that biological brains are Vision-Language Models. Our architecture is a \textit{demonstration}, not a \textit{mechanism}.
    \item A philosophical treatment of the Knowledge Argument or physicalism. However, we note that MaryVLM provides a unified framework for interpreting the various ``replies'' to Jackson's argument (see Discussion).
    \item A solution to the Hard Problem of consciousness. We do not explain why there is something it is like to be a system that processes information. No philosophical zombies were slain in the making of this paper.
\end{itemize}

\paragraph{This paper is:}
\begin{itemize}
    \item An empirical operationalization of Mary's Room using modern deep learning, with quantifiable metrics for ``surprise,'' ``learning,'' and ``subjectivity.''
    \item A demonstration that the properties commonly attributed to qualia---their apparent ineffability, privacy, intrinsicality, and direct apprehensibility---can emerge from the structure of information processing in a physical system.
    \item An argument that these properties are not mysterious extra ingredients, but expected consequences of architectural constraints.
\end{itemize}
