\section{Discussion}

Our findings suggest that the subjective character of qualia---the fact that sensory experiences feel like \textit{something}, and that my red might differ from yours---can be understood as a consequence of information processing architecture, not as a mysterious extra ingredient.

\subsection{Two Types of Novelty}

A central insight from our experiments is the distinction between \textbf{Sensory Novelty} and \textbf{Semantic Novelty}:

\begin{table*}[t]
    \centering
    \small
    \begin{tabular}{@{}lp{3.2cm}p{4cm}@{}}
        \hline
        \textbf{Type} & \textbf{Definition}           & \textbf{Example}                \\
        \hline
        Sensory       & Novel signal \textit{format}  & RGB to grayscale-trained system \\
        Semantic      & Novel signal \textit{content} & New object category             \\
        \hline
    \end{tabular}
    \caption{Sensory vs. Semantic Novelty. The ``Wow'' signal measures sensory novelty---surprise at format, not content.}
    \label{tab:novelty_types}
\end{table*}

Standard machine learning metrics (e.g., classification accuracy, perplexity) measure semantic novelty: does the model recognize the object? Our ``Wow'' signal measures something different---does the model recognize the \textit{format}? A grayscale-trained model can correctly identify a red apple (low semantic novelty) while still experiencing the chromatic format as profoundly unfamiliar (high sensory novelty). This dissociation is precisely what makes qualia puzzling: Mary knows everything about red, yet seeing it is still surprising.

\subsection{The Architecture of Subjectivity}

Our framework suggests that qualia emerge from a specific architectural pattern:

\begin{table*}[t]
    \centering
    \small
    \begin{tabular}{@{}lll@{}}
        \hline
        \textbf{Component} & \textbf{Name}         & \textbf{Role}                    \\
        \hline
        Vision Encoder     & Sensory Encoder       & Fixed hardware (format-specific) \\
        MLPs               & Conceptual Bridge     & Format translation interface     \\
        Language Model     & Propositional Decoder & ``Knowing that'' vs. experience  \\
        \hline
    \end{tabular}
    \caption{MaryVLM components and their philosophical roles.}
    \label{tab:architecture}
\end{table*}

The key constraint is the \textbf{asymmetry} between these components: the Sensory Encoder is frozen (phylogenetic), while the Conceptual Bridge and Propositional Decoder are plastic (ontogenetic). This asymmetry creates the conditions for qualia:

\begin{enumerate}
    \item The Sensory Encoder produces representations in a format determined by its fixed weights.
    \item The Conceptual Bridge learns to translate this format into language space---but only for formats it has seen during training.
    \item When a novel format arrives (chromatic input), the Bridge faces an unexpected translation cost---the ``Wow'' signal.
\end{enumerate}

Subjectivity arises because different agents (different random seeds) develop different Conceptual Bridges. They translate the same Sensory Encoding into different geometric arrangements in language space. The input is objective; the translation is subjective.

\subsection{Why Qualia Feel Private}

The Inverted Spectrum has long been cited as a puzzle for functionalism: how can two agents produce identical behaviors while having different internal experiences? Our results provide a computational demonstration:

\begin{itemize}
    \item Two MaryVLM agents with different random seeds achieve identical VQA accuracy (functional equivalence).
    \item Yet their internal representations of ``red'' occupy different geometric positions (structural disparity).
    \item The external function is shared; the internal geometry is private.
\end{itemize}

This is not a bug but a feature of learning systems. The Sensory Encoder provides a consistent input format, but the Conceptual Bridge has no constraint forcing it to match another agent's geometry. Each agent develops its own internal ``coordinate system'' for concepts. Communication succeeds not because agents share internal representations, but because they learn to map different internal representations to shared external symbols.

\subsection{Implications for the Knowledge Argument}

Our framework is agnostic on whether Mary learns a ``new fact'' or gains a ``new ability.'' Instead, we offer a more precise characterization: Mary's grayscale training creates a Conceptual Bridge optimized for achromatic-to-language translation. When chromatic input arrives:

\begin{enumerate}
    \item The Sensory Encoder produces a novel representation (it always encoded chromatic format---it was frozen).
    \item The Conceptual Bridge faces an Out-of-Distribution input, generating a high ``Wow'' signal.
    \item The Propositional Decoder receives a translation that does not match its learned distribution.
\end{enumerate}

Whether this constitutes ``new knowledge'' depends on how one defines knowledge. What our model demonstrates is that there is a \textit{measurable computational event} associated with Mary's first chromatic experience---an event that does not occur when she processes grayscale images of the same objects. The ``Hard Problem'' may remain, but the ``Wow'' signal is not mysterious; it is the cost of mismatched formats.

\subsection{Limitations}

Several limitations constrain our conclusions:

\begin{itemize}
    \item \textbf{Scale}: Our model is small (135M LLM parameters). Larger models may show different dynamics.
    \item \textbf{Simplicity of stimuli}: We used synthetic color-shape images. Natural images introduce confounds (textures, gradients) that may complicate interpretation.
    \item \textbf{Single modality}: We tested only vision. Generalizing to auditory or tactile qualia requires further work.
    \item \textbf{No temporal dynamics}: We measure a single ``Release'' event. Biological qualia adapt over time; our model does not capture habituation.
\end{itemize}

\subsection{Future Directions}

Several extensions suggest themselves:

\begin{enumerate}
    \item \textbf{Chromatic fine-tuning}: What happens to the ``Wow'' signal as Mary learns to process color? Does it decay as the Conceptual Bridge adapts?
    \item \textbf{Multi-modal qualia}: Extend to auditory stimuli (a grayscale-trained model exposed to audio) to test whether the ``Wow'' signal is modality-general.
    \item \textbf{Collective Mary}: Train multiple Marys with private experiences but a shared language, modeling how communities develop shared concepts despite private qualia.
    \item \textbf{Attention and gating}: Investigate whether the ``Wow'' signal modulates attention or ``promotes'' information to higher processing stages, as predicted by Global Workspace theory.
\end{enumerate}

