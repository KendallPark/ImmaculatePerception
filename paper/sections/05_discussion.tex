\section{Discussion}

Our \textit{MaryVLM} framework offers a rigorous computational translation of the ''Explanatory Gap,'' unifying philosophical insights with empirical metrics from cognitive science and AI. By modeling qualia not as ''raw feels'' but as the computational cost of format translation, we arrive at a physicalist account of subjectivity that respects the phenomenology of ''shock''.

\subsection{The Phenomenology of Prediction Error}

Our finding of a robust ''Wow'' signal unifies the philosophical concept of ''phenomenal surprise'' with the neuroscientific framework of \textit{Predictive Coding}. In biological systems, the Mismatch Negativity (MMN) signal arises when sensory input violates the brain's internal generative model. Our model replicates this: Mary’s ''shock'' is a massive Out-of-Distribution (OOD) error term generated when the ''read-only'' visual module forces a ''chromatic'' format into a ''luminance'' conceptual schema.

This suggests that the ''feeling'' of qualia is the metabolic and entropic cost of this representational friction. It aligns with the \textit{Global Neuronal Workspace} theory, which posits that only high-intensity prediction errors trigger ''global ignition'' and enter conscious awareness. In MaryVLM, the ''Wow'' signal is the threshold mechanism that promotes sensory data from unconscious processing to the conscious ''workspace'' of the language model.

\subsection{Resolving the ''Inverted Spectrum''}

\begin{itemize}
    \item Inverted Spectrum is often cited as a challenge to functionalism.
    \item We demonstrate, in physical terms, how the internal representations can be different, yet yield similar results.
    \item Tie in work by Kawakita (2025).
\end{itemize}

\subsection{The Limits of Encapsulation}

Our framework posits a strict architectural divide between fixed sensory encoding and plastic conceptual processing. However, to account for empirical evidence of perceptual learning, we must clarify how top-down signals interact with this divide without violating the ''read-only'' constraint. We propose a mechanism of \textbf{''Soft Encapsulation''} defined by the interaction between phylogenetic and ontogenetic systems.

\subsubsection{The Stability of Phylogenetic Representations (The ''Hard'' Boundary)}

The sensory encoder produces \textbf{Phylogenetic Representations}: high-dimensional states defined by evolutionarily fixed (frozen) weights. These states are \textbf{Cognitively Impenetrable}---they are ''read-only'' to the conceptual system. No amount of ''inner speech'' or conceptual belief can rewrite the vector coordinates of ''Red'' to match ''Green.'' The topology of the latent space is immutable.

\subsubsection{The Plasticity of Ontogenetic Representations (The ''Blurred'' Boundary)}

However, the downstream conceptual system operates on \textbf{Ontogenetic Representations}---malleable, ''read-write'' structures derived from social learning. While this system cannot \textit{overwrite} the Phylogenetic Representation, it can \textit{modulate} how it samples from it. We cite two well-documented effects as evidence for this mechanism:

\paragraph{Case A: Ontogenetic Modulation (The ''Russian Blue'' Effect)}
Research on categorical perception \citep{Wacongne2012} demonstrates that language speakers (e.g., Russian speakers distinguishing \textit{goluboy} from \textit{siniy}) exhibit faster discrimination of color boundaries. (Note: Reference likely needs update, checking Whorfian literature).

\textbf{Our Interpretation:} This is \textbf{Top-Down Attentional Gain}. The \textbf{Ontogenetic System} (Language) applies a ''predictive filter'' to the \textbf{Phylogenetic Representation} (Vision). The hardware format of ''blue'' remains Impenetrable (the rods and cones do not change), but the \textbf{Penetrable} conceptual layer learns to amplify specific dimensions of the sensory manifold. The experience changes not because the input was rewritten, but because the readout was sharpened.

\paragraph{Case B: Differentiation of Manifolds (The ''Wine Taster'' Effect)}
Similarly, perceptual learning literature \citep{Goldstone1998} shows that experts (e.g., wine tasters) perceive distinct features where novices perceive a unified ''blob.''

\textbf{Our Interpretation:} This represents the \textbf{Differentiation of Impenetrable States}. For a novice, the Ontogenetic system lacks the granularity to map the Phylogenetic input, resulting in a high-entropy ''Wow'' signal (confusion). For the expert, the Ontogenetic system has learned a higher-resolution mapping. The ''Wow'' signal decreases not because the sensory hardware changed (it is frozen), but because the \textbf{Software Resolution} of the conceptual system successfully differentiated the immutable hardware format.

\paragraph{Conclusion} The representation itself remains fixed (the hardware format is immutable), but the agent's sensitivity to it becomes refined. The ''Wow'' signal decreases over time because the \textbf{Ontogenetic System} successfully optimizes its interface with the \textbf{Phylogenetic} manifold, transitioning from ''Shock'' to ''Categorization.''

\subsection{Limitations and Future Directions}

Future work should extend this framework to auditory or tactile domains to test if the ''Wow'' signal is modality-invariant. Additionally, integrating this individual model into a \textit{Collective Predictive Coding} framework could demonstrate how a community of ''inverted'' Marys evolves a shared language despite their private representational differences.
